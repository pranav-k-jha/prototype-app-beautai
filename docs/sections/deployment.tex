\subsection{Mobile App Distribution}
\begin{itemize}[leftmargin=*]
    \item \textbf{App Store Deployment:}
    \begin{itemize}
        \item Use Expo's build services to generate platform-specific binaries (APK for Android, IPA for iOS).
        \item Follow guidelines for submitting apps to the Apple App Store and Google Play Store.
        \item Prepare necessary assets and metadata, such as app icons, screenshots, and descriptions.
    \end{itemize}
    
    \item \textbf{Versioning and Updates:}
    \begin{itemize}
        \item Implement a versioning strategy for releasing updates and new features.
        \item Use Expo's over-the-air (OTA) updates to push minor changes without requiring a full app store release.
        \item Ensure backward compatibility with previous app versions.
    \end{itemize}
    
    \item \textbf{Beta Testing:}
    \begin{itemize}
        \item Use TestFlight for iOS and Google Play's Internal Testing for Android to conduct beta testing.
        \item Gather feedback from beta testers to identify and fix issues before public release.
        \item Implement a feedback loop to continuously improve the app based on user input.
    \end{itemize}
\end{itemize}

\subsection{AWS Setup}
\begin{itemize}[leftmargin=*]
    \item \textbf{Infrastructure as Code:}
    \begin{itemize}
        \item Use Terraform or AWS CloudFormation to define and manage infrastructure.
        \item Version control infrastructure code alongside application code.
        \item Automate provisioning of AWS resources, including EC2, RDS, and S3.
    \end{itemize}
    
    \item \textbf{Compute Resources:}
    \begin{itemize}
        \item Deploy application containers using AWS ECS or EKS for orchestration.
        \item Configure auto-scaling groups to handle variable traffic loads.
        \item Use Elastic Load Balancers (ELB) for distributing incoming traffic.
    \end{itemize}
    
    \item \textbf{Database Setup:}
    \begin{itemize}
        \item Use Amazon RDS for MySQL as the primary database service.
        \item Enable automated backups and snapshots for data recovery.
        \item Configure read replicas for improved read performance and availability.
    \end{itemize}
\end{itemize}

\subsection{Domain Configuration}
\begin{itemize}[leftmargin=*]
    \item \textbf{Domain Registration:}
    \begin{itemize}
        \item Register domain names using AWS Route 53 or another domain registrar.
        \item Set up DNS records for mapping domain names to application endpoints.
    \end{itemize}
    
    \item \textbf{SSL/TLS Implementation:}
    \begin{itemize}
        \item Use AWS Certificate Manager to provision SSL/TLS certificates.
        \item Enforce HTTPS for all web traffic to ensure secure communication.
        \item Configure automatic certificate renewal to maintain security compliance.
    \end{itemize}
\end{itemize}

\subsection{CI/CD Pipeline}
\begin{itemize}[leftmargin=*]
    \item \textbf{Continuous Integration:}
    \begin{itemize}
        \item Use Jenkins, GitHub Actions, or AWS CodeBuild for continuous integration.
        \item Automate build and test processes for every code commit.
        \item Integrate code quality checks and linting into the CI pipeline.
    \end{itemize}
    
    \item \textbf{Continuous Deployment:}
    \begin{itemize}
        \item Use AWS CodeDeploy or Jenkins for automated deployment to staging and production environments.
        \item Implement blue/green or canary deployments to minimize downtime and risk.
        \item Rollback changes automatically in case of deployment failures.
    \end{itemize}
\end{itemize}

\subsection{Monitoring and Logging}
\begin{itemize}[leftmargin=*]
    \item \textbf{Monitoring Tools:}
    \begin{itemize}
        \item Use AWS CloudWatch for monitoring application performance and resource utilization.
        \item Set up alerts for critical metrics such as CPU usage, memory, and response times.
        \item Integrate with third-party monitoring tools like Datadog or New Relic for enhanced insights.
    \end{itemize}
    
    \item \textbf{Logging and Auditing:}
    \begin{itemize}
        \item Use AWS CloudWatch Logs for centralized logging of application events.
        \item Implement log rotation and retention policies to manage log data efficiently.
        \item Enable auditing of user actions and system changes for security compliance.
    \end{itemize}
\end{itemize}