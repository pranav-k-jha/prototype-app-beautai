\subsection{Version Control}
\begin{itemize}[leftmargin=*]
    \item \textbf{Git Workflow:}
    \begin{itemize}
        \item Use Git for version control, hosted on GitHub or GitLab.
        \item Follow a feature branch workflow to isolate development efforts.
        \item Use meaningful commit messages that describe changes clearly.
    \end{itemize}
    
    \item \textbf{Branching Strategy:}
    \begin{itemize}
        \item Maintain separate branches for development, staging, and production.
        \item Use feature branches for new features and bugfix branches for hotfixes.
        \item Merge changes into the main branch only after code review and testing.
    \end{itemize}
    
    \item \textbf{Pull Request Process:}
    \begin{itemize}
        \item Require pull requests for all changes to the main branch.
        \item Enforce code reviews with at least two approving reviewers.
        \item Use automated checks for code quality, linting, and testing.
    \end{itemize}
\end{itemize}

\subsection{Code Quality and Standards}
\begin{itemize}[leftmargin=*]
    \item \textbf{Coding Standards:}
    \begin{itemize}
        \item Follow language-specific style guides (e.g., PEP 8 for Python, AirBnB for JavaScript).
        \item Use consistent naming conventions for variables, functions, and classes.
        \item Write self-documenting code with clear and descriptive variable names.
    \end{itemize}
    
    \item \textbf{Code Quality Tools:}
    \begin{itemize}
        \item Use linters (e.g., ESLint, Pylint) to enforce coding standards.
        \item Integrate static analysis tools (e.g., SonarQube) for code quality assessment.
        \item Perform regular code refactoring to improve readability and maintainability.
    \end{itemize}
\end{itemize}

\subsection{Testing Requirements}
\begin{itemize}[leftmargin=*]
    \item \textbf{Unit Testing:}
    \begin{itemize}
        \item Achieve a minimum of 70\% unit test coverage for all modules.
        \item Write tests for all critical paths and edge cases.
        \item Use mocking frameworks to isolate unit tests from external dependencies.
    \end{itemize}
    
    \item \textbf{Integration Testing:}
    \begin{itemize}
        \item Test interactions between different components and services.
        \item Validate data flow and integration points with external systems.
        \item Use test doubles to simulate external service responses.
    \end{itemize}
    
    \item \textbf{End-to-End Testing:}
    \begin{itemize}
        \item Automate testing of user flows using tools like Cypress or Selenium.
        \item Ensure cross-browser compatibility and responsive design.
        \item Test performance and load handling under realistic scenarios.
    \end{itemize}
\end{itemize}

\subsection{Continuous Integration and Deployment}
\begin{itemize}[leftmargin=*]
    \item \textbf{CI/CD Pipeline:}
    \begin{itemize}
        \item Automate build, test, and deployment processes with CI/CD tools.
        \item Run automated tests on every code commit to catch regressions early.
        \item Deploy to staging environments for manual testing before production release.
    \end{itemize}
    
    \item \textbf{Environment Management:}
    \begin{itemize}
        \item Use environment variables to manage configuration settings.
        \item Separate configuration for development, staging, and production environments.
        \item Automate environment provisioning and teardown for consistency.
    \end{itemize}
\end{itemize}
